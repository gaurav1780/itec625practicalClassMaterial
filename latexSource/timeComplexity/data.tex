\input{itec625workshopHeader}
\section* {Learning outcomes}

By the end of this session, you will have learnt the basics about time complexity.

\section*{Questions}
\begin{questions}

\question What are the degrees of the following polynomials?

\begin{equation}
5x^3 + 3x - 7
\end{equation}

\ifprintanswers
ANSWER: 3
\else
\fi

\begin{equation}
5x - 2x^6
\end{equation}

\ifprintanswers
ANSWER: 6
\else
\fi

\question What are the time complexities of the following codes?

\begin{lstlisting}
int x = 1;
if(x % 2 == 0) {
	x++;
}
else {
	x--;
}
\end{lstlisting}

\ifprintanswers
ANSWER: $\mathcal{O}(1)$
\else
\fi

\begin{lstlisting}
for(int i=0; i < 100; i++) {
	System.out.print(i+ " ");
}
\end{lstlisting}

\ifprintanswers
ANSWER: $\mathcal{O}(1)$
\else
\fi


\begin{lstlisting}
for(int i=0; i < n; i++) {
	System.out.print(i+ " ");
}
\end{lstlisting}

\ifprintanswers
ANSWER: $\mathcal{O}(n)$
\else
\fi

\begin{lstlisting}
for(int i=0; i < n; i+=2) {
	if(i%3 == 0) {
		System.out.print(i+ ". ");
	}
	else {
		if(i%5 == 0) {
			System.out.print(i+ "! ");
		}
		else {
			System.out.print(i+ "? ");
		}
	}
}
\end{lstlisting}

\ifprintanswers
ANSWER: $\mathcal{O}(n)$
\else
\fi


\begin{lstlisting}
for(int i=1; i < n; i*=2) {
	System.out.print(i+ " ");
}
\end{lstlisting}

\ifprintanswers
ANSWER: $\mathcal{O}(log_2(n))$
\else
\fi

\begin{lstlisting}
for(int i=0; i < n; i+=n/5) {
	System.out.print(i+ " ");
}
\end{lstlisting}

\ifprintanswers
ANSWER: $\mathcal{O}(1)$
\else
\fi

\begin{lstlisting}
for(int i=6; i < n/2; i++) {
	System.out.print(i+ " ");
}
\end{lstlisting}

\ifprintanswers
ANSWER: $\mathcal{O}(n)$
\else
\fi


\begin{lstlisting}
for(int i=n/3; i < n/2; i+=4) {
	System.out.print(i+ " ");
}
\end{lstlisting}

\ifprintanswers
ANSWER: $\mathcal{O}(n)$
\else
\fi


\begin{lstlisting}
for(double i=1; i*i <=n; i++) {
	System.out.print(i+ " ");
}
\end{lstlisting}

\ifprintanswers
ANSWER: $\mathcal{O}(\sqrt{n})$
\else
\fi


\begin{lstlisting}
for(int i=0; i < n; i++) {
	for(int k=0; k < n; k++) {
		System.out.println(i+" ");
	}
}
\end{lstlisting}

\ifprintanswers
ANSWER: $\mathcal{O}(n^2)$
\else
\fi

\begin{lstlisting}
for(int i=0; i < n; i++) {
	for(int k=0; k < n; k++) {
		if(i%3 == 0) {
			System.out.print(i+ ". ");
		}
		else {
			if(i%5 == 0) {
				System.out.print(i+ "! ");
			}
			else {
				System.out.print(i+ "? ");
			}
		}
	}
}
\end{lstlisting}

\ifprintanswers
ANSWER: $\mathcal{O}(n^2)$
\else
\fi

\begin{lstlisting}
for(int i=n; i > 0; i-=2) {
	for(int k=1; k <= n; k+=3) {
		if(i%3 == 0) {
			System.out.print(i+ ". ");
		}
		else {
			if(i%5 == 0) {
				System.out.print(i+ "! ");
			}
			else {
				System.out.print(i+ "? ");
			}
		}
	}
}
\end{lstlisting}

\ifprintanswers
ANSWER: $\mathcal{O}(n^2)$
\else
\fi

\begin{lstlisting}
for(int i=0; i < n; i++) {
	for(int k=1; k <= n; k*=2) {
		if(i%3 == 0) {
			System.out.print(i+ ". ");
		}
		else {
			if(i%5 == 0) {
				System.out.print(i+ "! ");
			}
			else {
				System.out.print(i+ "? ");
			}
		}
	}
}
\end{lstlisting}

\ifprintanswers
ANSWER: $\mathcal{O}(n \times log_2(n))$
\else
\fi

\begin{lstlisting}
for(int i=1; i < n/2; i*=2) {
	for(int k=1; k < n/2; k*=2) {
		if(i%3 == 0) {
			System.out.print(i+ ". ");
		}
		else {
			if(i%5 == 0) {
				System.out.print(i+ "! ");
			}
			else {
				System.out.print(i+ "? ");
			}
		}
	}
}
\end{lstlisting}

\ifprintanswers
ANSWER: $\mathcal{O}((log_2(n))^2)$
\else
\fi

\begin{lstlisting}
for(double i=1; i<=n; i+=1.0/n) {
	System.out.print(i+ " ");
}
\end{lstlisting}


\ifprintanswers
ANSWER: $\mathcal{O}(n^2)$
\else
\fi


\question What is the time complexities for the method \texttt{foo} in each of the following codes?

\begin{lstlisting}
void foo(int n) {
	for(int i=0; i < n; i++) {
		System.out.print(i+ " ");
	}
}
\end{lstlisting}

\ifprintanswers
ANSWER: $\mathcal{O}(n)$
\else
\fi


\begin{lstlisting}
int foo(int n) {
	int result = 0;
	for(int i=0; i < n; i++) {
		result+=bar(n);
	}
	return result;
}

int bar(int n) {
	return n%2;
}
\end{lstlisting}

\ifprintanswers
ANSWER: $\mathcal{O}(n)$
\else
\fi

\begin{lstlisting}
int foo(int n) {
	int result = 0;
	for(int i=n; i > 0; i--) {
		result+=bar(i);
	}
	return result;
}

int bar(int n) {
	return n%2;
}
\end{lstlisting}

\ifprintanswers
ANSWER: $\mathcal{O}(n)$
\else
\fi

\begin{lstlisting}
int foo(int n) {
	int result = 0;
	for(int i=n; i > 0; i--) {
		result+=bar(i);
	}
	return result;
}

int bar(int n) {
	int total = 0;
	for(int i=1; i<=n; i+=2) {
		total+=i;
	}
	return total;
}
\end{lstlisting}

\ifprintanswers
ANSWER: $\mathcal{O}(n^2)$
\else
\fi

\begin{lstlisting}
int foo(int[] arr) {
	for(int i=1; i < arr.length; i++) {
		if(arr[i] < arr[i-1]) {
			return false;
		}
	}
	return true;
}
\end{lstlisting}

\ifprintanswers
ANSWER: 
Best case: $\mathcal{O}(1)$ (when first item is less than second item)
Worst case: $\mathcal{O}(n)$ (when each item is more than or equal to the item after it)
\else
\fi

\question Write a piece of code with $\mathcal{O}(log_2n)$ time complexity.

\ifprintanswers
ANSWER: 
\begin{lstlisting}
int total = 0;
for(int i=1; i<=n; i*=2) {
	total = total + i;
}
\end{lstlisting}

\else
\fi

\question Write a piece of code with a best case time complexity of $\mathcal{O}(n)$ and worst case time complexity of $\mathcal{O}(n^2)$.

\ifprintanswers
ANSWER: 
\begin{lstlisting}
int foo(int[] a) {
	for(int i=0; i < a.length; i++) {
		if(a[i]%2 == 0) { 
			for(int k=0; k < a.length; k++) {
				result+=a[k];
			}
		}
	}
	return result;
}
\end{lstlisting}
Best case $\mathcal{O}(n)$ when all numbers are odd
Worst case $\mathcal{O}(n^2)$ when all numbers are even
\else
\fi

\question Is it possible to have a code with a best case time complexity of $\mathcal{O}(n^2)$ and worst case time complexity of $\mathcal{O}(n*log_2n)$?

\ifprintanswers
ANSWER: 
No, best case time complexity cannot be worse than worst case time complexity.
\else
\fi

\end{questions}
\end{document}
