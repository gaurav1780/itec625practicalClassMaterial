\input{itec625workshopHeader}

\section*{Learning outcomes}

This weeks workshop aims at getting some practice with methods that operate on arrays. A template project is provided in \texttt{itec625workshop06template.zip}. No test file is provided but there is a client, whose output, when all methods are correctly implemented, should be:

\begin{verbatim}
90
null
null
100
0
2
2
5
[148, 184, 19, 65, 0]
[70, 80, 60]
\end{verbatim}

\newpage
\begin{questions}

\question \textbf{Method definition}

Define each of the following methods based on the specifications:

\begin{parts}
\part
\ifprintanswers
\begin{lstlisting}[style=junit]
/**
* @param arr
* @return the last item in the array.
* return null if array is null or empty
*/
public static Integer getLastItem(int[] arr) {
	if(arr == null || arr.length == 0)
		return null;
	return arr[arr.length - 1];
}
\end{lstlisting}	
\else
\begin{lstlisting}[style=junit]
/**
* @param arr
* @return the last item in the array.
* return null if array is null or empty
*/
public static Integer getLastItem(int[] arr) {








}
\end{lstlisting}	
\fi

\part
\ifprintanswers
\begin{lstlisting}[style=junit]
/**
* @param arr
* @param except: item to be excluded
* @return sum of all the items in the 
* array except item to be excluded
*/
public static int addAllBut(int[] arr, int except) {
	int total = 0;
	for(int i=0; i < arr.length; i++) {
		if(arr[i] != except) {
			total+=arr[i];
		}
	}
	return total;
}
\end{lstlisting}	
\else
\begin{lstlisting}[style=junit]
/**
* @param arr
* @param except: item to be excluded
* @return sum of all the items in the 
* array except item to be excluded
*/
public static int addAllBut(int[] arr, int except) {











}
\end{lstlisting}
\fi
\newpage
\part
\ifprintanswers
\begin{lstlisting}[style=junit]
/**
* @param arr
* @param start: starting index
* @param end: ending index 
* assume 0 <= start < arr.length
* assume 0 <= end < arr.length
* assume start <= end
* @return index of the smallest
* item in the index range [start, end]
*/
public static int getMinItemIndex(int[] arr, int start, int end) {
	int result = start;
	for(int i=start+1; i < end; i++) {
		if(arr[i] < arr[result]) {
			result = i;
		}
	}
	return result;
}
\end{lstlisting}	
\else
\begin{lstlisting}[style=junit]
/**
* @param arr
* @param start: starting index
* @param end: ending index 
* assume 0 <= start < arr.length
* assume 0 <= end < arr.length
* assume start <= end
* @return index of the smallest
* item in the index range [start, end]
*/
public static int getMinItemIndex(int[] arr, int start, int end) {









}
\end{lstlisting}
\fi

\part
\ifprintanswers
\begin{lstlisting}[style=junit]
/**
* @param a: assume every item occurs once
* @param b: assume every item occurs once
* @param c: assume every item occurs once
* @return number of items that exist
* in all three arrays
*/
public static int countCommonItems(int[] a, int[] b, int[] c) {
	int count = 0;
	for(int i=0; i < a.length; i++) {
		if(contains(b,a[i]) && contains(c,a[i])) {
			count++;
		}
	}
	return count;
}

//helper
public static boolean contains(int[] data, int item) {
	for(int i=0; i < data.length; i++) {
		if(data[i] == item) {
			return true;
		}
	}
	return false;
}
\end{lstlisting}	
\else
\begin{lstlisting}[style=junit]
/**
* @param a: assume every item occurs once
* @param b: assume every item occurs once
* @param c: assume every item occurs once
* @return number of items that exist
* in all three arrays
*/
public static int countCommonItems(int[] a, int[] b, int[] c) {










}

//HINT: Write a helper method that
//returns true if an array contains 
//a given item, false otherwise












\end{lstlisting}
\fi

\newpage

\part
\ifprintanswers
\begin{lstlisting}[style=junit]
/**
* @param source
* @param idx: index of item in array
* source (assume 0 <= idx < source.length)
* @param dest
* @return index (in array dest) of
* item at index idx (in array source).
* return -1 if item doesn't exist in dest
*/
public static int vlookup(int[] a, int idx, int[] b) {
	for(int i=0; i < b.length; i++) {
		if(b[i] == a[idx]) {
			return i;
		}
	}
	return -1;
}
\end{lstlisting}	
\else
\begin{lstlisting}[style=junit]
/**
* @param source
* @param idx: index of item in array
* source (assume 0 <= idx < source.length)
* @param dest
* @return index (in array dest) of
* item at index idx (in array source).
* return -1 if item doesn't exist in dest
*/
public static int vlookup(int[] a, int idx, int[] b) {











}
\end{lstlisting}
\fi

\part
\ifprintanswers
\begin{lstlisting}[style=junit]
/**
* @param data
* @param nBits, assume nBits.length == data.length
* modify the array data such that each
* item is left shifted by 
* corresponding number of bits from
* array nBits
* NOTE: assume each item of nBits is non-negative
*/
public static void leftShift(int[] data, int[] nBits) {
	for(int i=0; i < data.length; i++) {
		data[i] = data[i] << nBits[i];
	}
}
\end{lstlisting}	
\else
\begin{lstlisting}[style=junit]
/**
* @param data
* @param nBits, assume nBits.length == data.length
* modify the array data such that each
* item is left shifted by 
* corresponding number of bits from
* array nBits
* NOTE: assume each item of nBits is non-negative
*/
public static void leftShift(int[] data, int[] nBits) {






}
\end{lstlisting}
\fi

\newpage

\part \textbf{(Advanced)}
\ifprintanswers
\begin{lstlisting}[style=junit]
/**
* @param a: assume every item occurs once
* @param b: assume every item occurs once
* @param c: assume every item occurs once
* @return: array containing items that
* occur in exactly two of the three arrays
*/
public static int[] twoOutOfThree(int[] a, int[] b, int[] c) {
	int count = 0;
	for(int i=0; i < a.length; i++)
		if(contains(b, a[i]) != contains(c, a[i]))
			count++;
	for(int i=0; i < b.length; i++)
		if(contains(c, b[i]) && !contains(a, b[i]))
			count++;

	int[] result = new int[count];
	
	int k = 0;
	for(int i=0; i < a.length; i++)
		if(contains(b, a[i]) != contains(c, a[i]))
			result[k++] = a[i];
	for(int i=0; i < b.length; i++)
		if(contains(c, b[i]) && !contains(a, b[i]))
			result[k++] = b[i];

	return result;
}

//helper
public static boolean contains(int[] data, int item) {
	for(int i=0; i < data.length; i++) {
		if(data[i] == item) {
			return true;
		}
	}
	return false;
}
\end{lstlisting}	
\else
\begin{lstlisting}[style=junit]
/**
* @param a: assume every item occurs once
* @param b: assume every item occurs once
* @param c: assume every item occurs once
* @return: array containing items that
* occur in exactly two of the three arrays
*/
public static int[] twoOutOfThree(int[] a, int[] b, int[] c) {











}
\end{lstlisting}
\fi
\end{parts}

\end{questions}	
\end{document}
