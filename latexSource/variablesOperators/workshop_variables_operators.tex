\documentclass{exam} 
\def\workshopTitle{Workshop - Variables and Operators} 
\input{itec625workshopHeader} 
\section*{Learning outcomes}

This weeks workshop is about understanding variables and operators.

\vspace{1em}
\begin{questions}

\question \textbf{Twos complement}

Convert the following negative decimal numbers to binary assuming storage is in 1 byte and the first bit is used for sign:

\begin{enumerate}
\item -50
\item -8
\item -99
\end{enumerate}

\begin{solution}
\begin{enumerate}
\item -50 = 11001110
\item -8  = 11111000
\item -99 = 10011101
\end{enumerate}	
\end{solution}

\question \textbf{Storage}
\vskip 0.5cm

Can the following values be stored correctly in the data type assigned to them? If so, state the complete bit-pattern that the following variables are stored as, in the memory.

\begin{lstlisting}
short b = 28;
int c = 53;
byte a = 130;
long d = 65536;
short e = -203;
\end{lstlisting}

\begin{solution}
\begin{lstlisting}
short b = 28; //valid, 00000000 00011100
int c = 53; 
//valid, 00000000 00000000 00000000 00110101
byte a = 130; //invalid
long d = 65536; 
//valid, 00000000 00000000 00000000 00000000 
//       00000000 00000001 00000000 00000000
short e = -203; //valid, 11111111 00110101
\end{lstlisting}	
\end{solution}

\question \textbf{Number systems}
\vskip 0.5cm

Convert each of the following decimal numbers into binary, quinary (base-5) and nonary (base-9)

\begin{enumerate}
\item 8	
\item 29
\item 52
\end{enumerate}

\begin{solution}
	
\begin{enumerate}
\item 8	in binary: 1000, in quinary: 13, in nonary: 8 
\item 29 in binary: 11101, in quinary: 104, in nonary: 32 
\item 52 in binary: 110100, in quinary: 202, in nonary: 57 
\end{enumerate}
\end{solution}

\question \textbf{Converting to power of source base}

In the lectures, we demonstrated a way to convert an integer $n$ from base-$p$ to base-$q$ when $q = p^k$ ($k$ being an integer more than 1).

\begin{verbatim}
EXAMPLE 1:

n = 11101010, p = 2, q = 8 
q = 2^3, therefore, k = 3
Split n in groups of 3 (starting from right side)

11 101 010

Pad the first group if incomplete with 0s.

011 101 010

Convert each group to decimal individually.

3 5 2

Put it together. That's the number in base 8.

Hence, 11101010 in base-2 = 352 in base-8.
\end{verbatim}

\newpage

\begin{verbatim}
EXAMPLE 2:

n = 10201221
p = 3
q = 9
q = 3^2, therefore, k = 2
Split n in groups of 2 (starting from right side)

10 20 12 21

Pad the first group if incomplete with 0s - not applicable

Convert each group to decimal individually.

3 6 5 7

Put it together. That's the number in base 9.

Hence, 10201221 in base-3 = 3657 in base-9.
\end{verbatim}


Convert the following numbers (source and destination bases provided):

\begin{enumerate}
\item 11100010 in base-2 to base-4	
\item 11100010 in base-2 to base-8
\item 11100010 in base-2 to base-16	
\item 120100121 in base-3 to base-9
\item 310223201 in base-4 to base-16
\end{enumerate}

\begin{solution}
\begin{enumerate}
\item 11100010 in base-2 = 3202 in base-4	
\item 11100010 in base-2 = 342 in base-8
\item 11100010 in base-2 = e2 in base-16	
\item 120100121 in base-3 = 16317 in base-9
\item 310223201 in base-4 = 34ae1 base-16
\end{enumerate}	
\end{solution}

\question \textbf{Converting from power of destination base}
\vskip 0.5cm

Now we'll do the opposite - convert an integer $n$ from base-$q$ to base-$p$ when $q = p^k$ ($k$ being an integer more than 1).

\begin{verbatim}
EXAMPLE 1:

n = e8f2
q = 16
p = 2
q = 2^4, therefore, k = 4

Convert each symbol to decimal and then to base p.

e      8     f      2
14     8     15     2
1110   1000  1111   10

Pad with leading zeroes to make groups of size k (4)

1110   1000  1111   0010

Put it together

1110100011110010

That's your number in base-p (2)

Hence, e8f2 in base-16 = 1110100011110010 in base-2.
\end{verbatim}

Convert the following numbers (source and destination bases provided):

\begin{enumerate}
\item 5073 in base-9 to base-3
\item abc123 in base-16 to base-2
\end{enumerate}

\begin{solution}
\begin{enumerate}
\item 5073 in base-9 = 12002110 in base-3
\item abc123 in base-16 = 10101011 11000001 00100011 base-2
\end{enumerate}
\end{solution}

\question \textbf{Expressions}
\vskip 0.5cm

An expression is an operation evaluating to a specific value.

What are the values of the following arithmetic expressions?

\begin{enumerate}
\item 17/5
\item 1.0 + 16/5
\item (1.0 + 16)/5
\item 3 * ((2 + 5) / (4 - 1) + 17 \% 5)
\end{enumerate}

\begin{solution}
\begin{enumerate}
\item 17/5 = 3
\item 1.0 + 16/5 = 4.0
\item (1.0 + 16)/5 = 3.4
\item 3 * ((2 + 5) / (4 - 1) + 17 \% 5) = 12
\end{enumerate}	
\end{solution}

\question \textbf{Boolean expressions}
\vskip 0.5cm

What are the values of the following boolean expressions?

\begin{enumerate}
\item true \&\& false
\item true \&\& (false || true)
\item true || false
\item false || !(true || false)
\item (5 >= 0 \&\& (5 <= 2 || 5 <= 10))
\end{enumerate}

\begin{solution}
\begin{enumerate}
\item true \&\& false = false
\item true \&\& (false || true) = true
\item true || false = true
\item false || !(true || false) = false
\item (5 >= 0 \&\& (5 <= 2 || 5 <= 10)) = true
\end{enumerate}	 
\end{solution}


\question \textbf{Bitwise operations}
\vskip 0.5cm

An expression is an operation evaluating to a specific value.

What are the values of the following bitwise operations?

\begin{enumerate}
\item 21 \& 19
\item 21 $|$ 19
\item 21 \^{} 19
\item 12 $<<$ 2
\item 12 $>>$ 2
\end{enumerate}

\begin{solution}
\begin{enumerate}
\item 21 \& 19 = 17
\item 21 $|$ 19 = 23
\item 21 \^{} 19 = 6
\item 12 $<<$ 2 = 48
\item 12 $>>$ 2 = 3
\end{enumerate}	
\end{solution}

\question \textbf{Java program}
\vskip 0.5cm

Consider the following scenario:

\textit{John takes 5 hours to paint 3 square meters while Jenny takes 15 hours to paint 7 square meters.}

If they work together, how much time will they need to paint a wall whose area is 56 square meters? Also determine how much area John paints and how much area Jenny paints.

First devise a solution on paper and verify your solution by ensuring that:

\begin{verbatim}
(area painted by john in determined time) +
(area painted by jenny in the same determined time) = total area (56)
\end{verbatim}

Then, write a java program in Eclipse that solves the following problem. 

IMPORTANT: Think about the types you'll use to store these values?

\begin{solution}

\begin{enumerate}
	\item Calculate area that John can paint in an hour and area that Jenny can paint in an hour.
	\item Calculate he total area they can paint in an hour working together.	
	\item Using this and the total area that needs to be painted, compute the number of hours they should work together.
	\item Calculate area painted by John and Jenny individually in that time.
\end{enumerate}

\lstinputlisting[language=java, basicstyle=\small]{Painting.java}
\end{solution}


\end{questions}
\end{document}
